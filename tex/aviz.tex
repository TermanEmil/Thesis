\begin{titlepage}

    \newcommand{\HRule}{\rule{\linewidth}{0.5mm}}
    
    \begin{center}
        \textbf{
            Universitatea Tehnică a Moldovei\\
            Facultatea Calculatoare, Informatică și Microelectronică\\
            Departamentul Ingineria Software și Automatică\\
            Programul de studii Tehnologii Informaționale\\
        }
    \end{center}
    \center
    
    \vspace{0.3cm}
    \begin{center}
        \large \textbf{AVIZ} \\
        Teza de licenţă \\
    \vspace{0.2cm}
    
    \end{center}
    
    \begin{flushleft}
        \textbf{Tema:} \ThesisTitle \\
        \textbf{Studentul:} \MyNameFull
        
        \begin{enumerate}[leftmargin=1.4em, itemsep=1pt, parsep=0pt]
        \item [1.] \textbf{Caracteristica tezei de licență:}  Teza a fost elaborată în conformitate cu toate cerințele și standardele în vigoare; reprezintă o analiză detaliată a implementărilor existente, depășind domeniul imediat al implementării soluției propuse și analizînd scenarii similare în alte domenii. Teza reprezintă un vestigiu coerent al lucrului intelectual efectuat de către student.
        
        \item [2.] \textbf{Estimarea rezultatelor obținute:} Rezultatele obținute sunt deja aplicabile; aplicația e bine documentată, iar codul este deloc criptic, bine organizat și elaborat cu responsabilitate.
        
        \item [3.] \textbf{Corectitudinea materialului expus:} Referințele externe au fost luate din surse respectabile, iar sinteza informației obținute în rezultatul lucrului și a evaluării surselor externe este concludentă și logică. Teza corespunde standardelor tehnice, inclusiv analiza economică a proiectului și diagramele utilizate. 
        
        \item [4.] \textbf{Calitatea materialului grafic:} Figurile prezentate în lucrarea dată descriu si analizeaza în detaliu sistemul curent. Diagramele UML corespund standartului 2.0. Materialul grafic prezentat în lucrare îşi îndeplineşte scopul de a îmbunătăţi descrierea sistemului impementat.
        
        \item [5.] \textbf{Observații și recomandări:} Această teză se face evidentă, în primul rînd, datorită caracterului practic al problemei pe care o abordează. Este recomandat de a extinde sistemul curent cu noi funcţionalităţi.
        
        \item [6.] \textbf{Caracteristica studentului și titlul conferit:} Studentul a demonstrat o asiduitate deosebită în studierea independentă a unei teme netriviale și o abordare inginerească în soluționarea problemei specificate.
        
        \textbf{Rezultatele obținute în cadrul tezei îmi permit să recomand admiterea tezei de licență d-lui \MyName spre susținere și să o apreciez cu nota maximă.}
        
        \end{enumerate}
        
        \hfill Conducătorul tezei de licență\\
        \hfill \textbf{\Coordonator}\\

    \end{flushleft}
    \vfill
\end{titlepage}