\section*{Rezumat}

Lucrarea \textbf{DSLR Camera Controller} prezentată de Terman Emil a fost scrisă a fost scrisă în engleză. Ea constă din 25 de figuri, 11 secvențe de cod, 8 tabele și 15 referințe. Lucrarea face parte din introducere, 4 capitole și concluzii.

Teza are ca scop sa dezvolte un instrument pentru a configura si controla de la distanta camerile DSLR. Aceasta este of platform care in mare parte se focuseaza pentru a rezolva problema de luare repetata a fotografiilor intr-o perioada lunga de timp (unul-doi ani sau mai mult).

Applicatia a fost dezvoltată utilizând limbajul de programare Python. Instrumentul consită din doua parti: comunicarea cu cameri si clientul. Tehnologiile utilizate sunt frameworkul Django pentru a dezvolta API-ul web, SQL Server pentru a stoca datele, GPhoto2 pentru comunicarea cu cameri si Ykush pentru a reseta connectiunea cu cameri.

DSLR Camera Controller reprezintă un instrument pentru fotografi care vreau sa capteze fotografii time-lapsuri intr-o manera foarte sigura. Acesta ofera optiunele de a modifica setarile camerilor, de a configura un transferul de imagini, de a alege o strategie in cazul in care camerile intalnesc o eroare si programarea time-lapsurilor.